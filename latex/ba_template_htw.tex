% Name und persönliche Daten nur auf DECKBLATT und EIDESSTATTLICHER ERKLÄRUNG
% DECKBLATT, SPERRVERMERK, VORWORT ohne Seitenzahlen
% INHALTSVERZ., AUFGABENSTELLUNG, VERZEICHNISSE römische Nummerierung (I, II, III)
% ALLES ANDERE arabisch (1, 2, 3)

% DIN A4
% Oben 		= 2,5 cm
% Unten 	= 2,5 cm
% Links 	= 3,0 cm
% Rechts 	= 2,5 cm

% Arial 11pt, Times New Roman 12pt
% Zeilenabstand: einfach -1.5
% Blocksatz mit Silbentrennung
% Überschriften im Text abheben (fett oder so)
% Zitierstil: IEEE
% Bild 1: <Beschreibung> (unten)
% Tabelle 1: <Beschreibung> (oben)
% caption links und label in caption Fett!


%%%%%%%%%%%%%%%%%%%%%%%%%%%%%%%%%%%%%
%%% BIBLIOTHEKEN %%%%%%%%%%%%%%%%%%%%
%%%%%%%%%%%%%%%%%%%%%%%%%%%%%%%%%%%%%

\documentclass[a4paper,12pt,headsepline]{scrartcl}

\usepackage[utf8]{inputenc} 	% utf-8 encoding
\usepackage{graphicx} 			% Grafiken
\usepackage[ngerman]{babel} 	% deutsch als Hauptsprache
\usepackage[right]{eurosym} 	% Eurozeichen
\usepackage{fontspec} 			% Zeichenencoding
\usepackage{lmodern}			% flüssigere Schrift
\usepackage{longtable} 			% mehrseitige Tabellen
\usepackage[left=3cm,right=2.5cm,top=2.5cm,bottom=2.5cm,includeheadfoot]{geometry}
								% Seitenränder
\usepackage{fancybox} 			% Paket für Boxen im Text
\usepackage{url} 				% bricht lange URLs "schön" um
\usepackage[dvipsnames]{xcolor}	% Textfarben (dvipsnames = mehr farben!)
\usepackage{amsmath,amsthm,amssymb,amsfonts}
								% Mathe Formeln, Zeichen etc.
\usepackage{hyperref} 			% erzeugt Inhaltsverzeichnis mit Querverweisen zu den Abschnitten (PDF Version)
\usepackage{fancyhdr} 			% Kopf- Fußzeilen
%\usepackage{array} 			% für Tabellen
\usepackage[onehalfspacing]{setspace} 
								% Paket für Zeilenabstand
\usepackage{caption}			% extended captions (fettes label)
\usepackage{makeidx} 			% extended index
\usepackage{listings} 			% code Blöcke
\usepackage[backend=biber, sorting=none, citestyle=ieee]{biblatex}
%\usepackage{biblatex}

\usepackage[german]{nomencl}
\usepackage{acronym}
\usepackage{tabularx}
%\usepackage{blindtext}
%\usepackage{csquotes}
\usepackage{multirow}
\usepackage{float}				% Bilder mit [H] genau dort einordnen wo sie hin sollen
%\usepackage{dblfloatfix}
%\usepackage[titles]{tocloft} 	% Für Formeln und Formelverzeichnis
%\usepackage{floatflt} 			% floatende Bilder ermöglichen
\usepackage{booktabs}			% Toprule etc.
\usepackage{subcaption}			% Bilder nebeneinander
\usepackage{tocloft}			% for cftdot command

%%%%%%%%%%%%%%%%%%%%%%%%%%%%%%%%%%%%%
%%% EINSTELLUNGEN %%%%%%%%%%%%%%%%%%%
%%%%%%%%%%%%%%%%%%%%%%%%%%%%%%%%%%%%%

%% no dots in toc
%\renewcommand{\cftdot}{}

%% for dots
\renewcommand{\cftsecleader}{\cftdotfill{\cftdotsep}} % for sections
%\renewcommand{\cftchapleader}{\cftdotfill{\cftdotsep}} % for chapters
%\renewcommand{\cftpartleader}{\cftdotfill{\cftdotsep}} % for parts

%% Sektion Einstellungen
%\titleformat{\section}
%{\normalfont\sffamily\huge\bfseries\color{black}}
%{\thesection}{20pt}{\Huge}
%\titleformat{\subsection}
%{\normalfont\sffamily\huge\bfseries\color{black}}
%{\thesubsection}{20pt}{\Huge}

%% Sektions gleicher Font wie Text
\addtokomafont{disposition}{\rmfamily}
\setkomafont{title}{\huge\bfseries}
\addtokomafont{titlehead}{\LARGE\centering}

% label in captions fett, caption links aligned! 
\captionsetup{labelfont=bf, justification=raggedright, singlelinecheck=false}

% Pfad für bilder setzen
%\graphicspath{./abb/}

% Neuer Absatz ohne Ausrückung
\setlength{\parindent}{0pt}
%\geometry{left=3cm, right=2.5cm, top=2.5cm, bottom=2.5cm}

% Farben definieren für CODE
\definecolor{codeGreenDark}{HTML}{1f9109}
\definecolor{codeGreenMint}{HTML}{00e061}
\definecolor{codeGray}{HTML}{a3a3a3}
\definecolor{codePurple}{HTML}{c247ff}
\definecolor{codeOrange}{HTML}{e39b00}
\definecolor{codeDarkGray}{HTML}{303030}

% Farben definieren für TEXT
\definecolor{txtRed}{HTML}{ff0000}

% Basic Setup
\newcommand{\codeBg}{\color{codeDarkGray}}
\newcommand{\codeText}{\color{white}}
\newcommand{\codeTextKeyword}{\color{codePurple}}
\newcommand{\codeTextStrings}{\color{codeOrange}}
\newcommand{\codeTextComment}{\color{codeGray}}
\newcommand{\codeTextLineNmbrs}{\color{codeDarkGray}}
\newcommand{\codeFrame}{\color{codeDarkGray}}

%%%%%%%%%%%%%%%%%%%%%%%%%%%%%%%%%%%%%
%%% SYMBOLVERZEICHNIS %%%%%%%%%%%%%%%
%%%%%%%%%%%%%%%%%%%%%%%%%%%%%%%%%%%%%

% Punkte zw. Abkürzung und Erklärung
\setlength{\nomlabelwidth}{.20\hsize}
\renewcommand{\nomlabel}[1]{#1 \dotfill}
% Zeilenabstände verkleinern
\setlength{\nomitemsep}{-\parsep}

%%%%%%%%%%%%%%%%%%%%%%%%%%%%%%%%%%%%%
%%% LISTINGS/CODE SETUP %%%%%%%%%%%%%
%%%%%%%%%%%%%%%%%%%%%%%%%%%%%%%%%%%%%

\lstset{ 
	backgroundcolor=\codeBg, 	% choose the background color; you must add \usepackage{color} or \usepackage{xcolor}; should come as last argument
	basicstyle=\footnotesize\ttfamily\codeText,      	% the size of the fonts that are used for the code
	breakatwhitespace=false,       	% sets if automatic breaks should only happen at whitespace
	breaklines=true,               	% sets automatic line breaking
	captionpos=b,                  	% sets the caption-position to bottom
	commentstyle=\codeTextComment,  	% comment style
	deletekeywords={...},          	% if you want to delete keywords from the given language
	escapeinside={\%*}{*)},        	% if you want to add LaTeX within your code
	extendedchars=true,            	% lets you use non-ASCII characters; for 8-bits encodings only, does not work with UTF-8
	firstnumber=1,              	% start line enumeration with line 1000
	%frame=single,	               	% adds a frame around the code
	keepspaces=true,               	% keeps spaces in text, useful for keeping indentation of code (possibly needs columns=flexible)
	keywordstyle=\color{blue},     	% keyword style
	language=Octave,               	% the language of the code
	morekeywords={*,...},          	% if you want to add more keywords to the set
	numbers=left,                  	% where to put the line-numbers; possible values are (none, left, right)
	numbersep=5pt,                 	% how far the line-numbers are from the code
	numberstyle=\codeTextLineNmbrs,% the style that is used for the line-numbers
	%numberstyle=\tiny\color{codeGray},% the style that is used for the line-numbers
	rulecolor=\codeFrame,       	% if not set, the frame-color may be changed on line-breaks within not-black text (e.g. comments (green here))
	showspaces=false,              	% show spaces everywhere adding particular underscores; it overrides 'showstringspaces'
	showstringspaces=false,        	% underline spaces within strings only
	showtabs=false,                	% show tabs within strings adding particular underscores
	stepnumber=1,                  	% the step between two line-numbers. If it's 1, each line will be numbered
	stringstyle=\ttfamily\color{codePurple},   	% string literal style
	tabsize=4,	                   	% sets default tabsize to 2 spaces
	title=\lstname                 	% show the filename of files included with \lstinputlisting; also try caption instead of title
}

%%\usepackage{beramono}
% Style definition for languages
\lstdefinestyle{customC}{
	belowcaptionskip=1\baselineskip,
	breaklines=true,
	frame=single,
	%xleftmargin=\parindent,
	language=C,
	showstringspaces=false,
	%basicstyle=\footnotesize\ttfamily,
	keywordstyle=\bfseries\codeTextKeyword,
	commentstyle=\itshape\codeTextComment,
	%identifierstyle=\color{codeGreenMint},
	stringstyle=\ttfamily\codeTextStrings,
}

\lstdefinestyle{customPy}{
	belowcaptionskip=1\baselineskip,
	breaklines=true,
	frame=single,
	%xleftmargin=\parindent,
	language=python,
	showstringspaces=false,
	%basicstyle=\footnotesize\ttfamily,
	keywordstyle=\bfseries\codeTextKeyword,
	commentstyle=\itshape\codeTextComment,
	%identifierstyle=\color{codeGreenMint},
	stringstyle=\ttfamily\codeTextStrings,
}

%% eigene Kopfzeilen und Fußzeilen
\fancypagestyle{contents}{%
	\fancyhf{} 								% alle Kopf- und Fußzeilenfelder bereinigen
	\fancyhead[L]{\nouppercase{\leftmark}} 	% Kopfzeile links
	\fancyhead[C]{} 						% zentrierte Kopfzeile
	\fancyhead[R]{\thepage} 				% Kopfzeile rechts
	\renewcommand{\headrulewidth}{0.4pt} 	% obere Trennlinie
} % eigener Seitenstil für Inhalte

\fancypagestyle{preamble}{%
	\fancyhf{} 								% alle Kopf- und Fußzeilenfelder bereinigen
	\fancyhead{} 							% keine Kopfzeile
	\fancyfoot[C]{\thepage} 				% Kopfzeile rechts
	\renewcommand{\headrulewidth}{0pt} 	% obere Trennlinie
} % eigener Seitenstil für Preambles

%%\bibliographystyle{IEEEtran}
% Bibliographie
\addbibresource{ref.bib}
\bibliography{ref}

% Inhaltsverzeichnis
\makeindex

% Abkürzungsverzeichnis LiveTex Version
\makenomenclature

%% table of contents setup
%\setlength\cftparskip{\tocspacing}
\setlength\cftbeforesecskip{5pt}
%\setlength\cftaftertoctitleskip{2pt}

%\newcommand{\indexesspacing}{\cftbeforesecskip}
\newcommand{\indexesspacing}{\cftbeforesecskip}

\setlength\cftbeforesubsecskip{\indexesspacing}
\setlength\cftbeforesubsubsecskip{\indexesspacing}
\setlength\cftbeforefigskip{\indexesspacing}
%\setlength\cftbeforesubfigskip{\indexesspacing}
\setlength\cftbeforetabskip{\indexesspacing}
%\setlength\cftbeforesubtabskip{\indexesspacing}

\begin{document}
	% defined commands and variables

\newcommand{\THEMA}{dies ist mein Thema}

\newcommand{\inquote}[1]{
	\glqq{#1}\grqq{}
}

%%% LABEL ALWAYS AFTER CAPTION %%%
\newcommand{\image}[4][]{
	\begin{figure}[H]
		%
		\centering
		%\includegraphics[width=1\textwidth]{images/RIFB.png}
		%\includegraphics[height=10cm, keepaspectratio=true]{#1}
		\includegraphics[width=\ifthenelse{\equal{#1}{}}{1\textwidth}{#1}, keepaspectratio=true]{#2}
		\caption{#4}
		\label{#3}
	\end{figure}
}

\newcommand{\doublefigure}[6]{
	\begin{figure}[H]
		\centering
		\begin{subfigure}[b]{.5\textwidth}
			\centering
			%\includegraphics[width=0.9\textwidth]{#1}
			\includegraphics[height=10cm]{#1}
			\caption{#2}
		\end{subfigure}%
		\begin{subfigure}[b]{.5\textwidth}
			\centering
			%\includegraphics[width=0.9\textwidth]{#3}
			\includegraphics[height=10cm]{#3}
			\caption{#4}
		\end{subfigure}
		\caption{#5}
		\label{#6}
	\end{figure}
}

\newcommand{\newparagraph}{\\[12pt]}
	
	%%%%%%%%%%%%%%%%%%%%%%%%%%%%%%%%%%%%%
	%%% DECKBLATT %%%%%%%%%%%%%%%%%%%%%%%
	%%%%%%%%%%%%%%%%%%%%%%%%%%%%%%%%%%%%%
	
	\pagestyle{empty}
	
	\singlespacing
	\thispagestyle{empty}


\begin{figure}[t]
 \centering
 \includegraphics[width=0.3\textwidth]{abb/logo_htw.png}
\end{figure}

\begin{verbatim}
\end{verbatim}

\begin{center}
\Large{Hochschule für Technik und Wirtschaft Berlin}\\
\end{center}

\begin{center}
\Large{Fachbereich für Ingenieurwissenschaften - Technik und Leben}
\end{center}
\begin{verbatim}
\end{verbatim}
\begin{center}
\doublespacing
\textbf{\LARGE{Bachelorarbeit}}\\
\singlespacing
\begin{verbatim}
\end{verbatim}
\textbf{{im Studiengang Ingenieurinformatik}}
\end{center}
\begin{verbatim}
\end{verbatim}
\begin{center}

\end{center}
\begin{verbatim}
\end{verbatim}
\begin{flushleft}
\begin{tabular}{llll}
\textbf{Thema:} & & <Thema> & \\
& & \\
\textbf{Autor:} & & Tworkowski, Oliver& \\
& & \\
\textbf{Version vom:} & & \today &\\
& & \\
\textbf{Betreuer:} & & <Betreuer> &\\
% \textbf{2. Betreuer:} & & Prof. Dr. Y &\\
\end{tabular}
\end{flushleft}

	\newpage
	\onehalfspacing
	
	\textcolor{gray}{\textit{Diese Seite wurde absichtlich frei gelassen}}
	\newpage
	
	\section*{Vorwort}

Ich grüße meine Tante aus Wuppertal
	\newpage
	
	% Sperrvermerk
	% \input{sperrvermerk}
	% \newpage
	
	%%%%%%%%%%%%%%%%%%%%%%%%%%%%%%%%%%%%%
	%%% VERZEICHNISSE %%%%%%%%%%%%%%%%%%%
	%%%%%%%%%%%%%%%%%%%%%%%%%%%%%%%%%%%%%
	
	\pagestyle{preamble}
	
	% Inhaltsverzeichnis
	\setcounter{page}{1}
	\pagenumbering{Roman}
	\tableofcontents
	\newpage
	
	% Abbildungsverzeichnis
	\addcontentsline{toc}{section}{Abbildungsverzeichnis}
	\listoffigures
	\newpage
	
	% Tabellenverzeichnis
	\addcontentsline{toc}{section}{Tabellenverzeichnis}
	\listoftables
	\newpage
	
	%	% Formelverzeichnis
	%	\addcontentsline{toc}{section}{Formelverzeichnis}
	%	\newcommand{\listequationsname}{Formelverzeichnis}
	%	\newlistof{myequations}{equ}{\listequationsname}
	%	\newcommand{\myequations}[1]{%
	%		\addcontentsline{equ}{myequations}{\protect\numberline{\theequation}#1}\par}
	%	\listofmyequations
	%	\newpage
	
	% Quellcodeausschnittverzeichnis
	\renewcommand\lstlistingname{Quellcodeausschnitt}
	\renewcommand\lstlistlistingname{Quellcodeausschnittverzeichnis}
	\addcontentsline{toc}{section}{Quellcodeausschnittverzeichnis}
	\lstlistoflistings
	\newpage
	
	% Abkürzungsverzeichnis
	\section*{Abkürzungsverzeichnis}\label{abkuerzung}
	\addcontentsline{toc}{section}{Abkürzungsverzeichnis}
	% \acro{kürzel}[kurzform]{langform}
	\begin{acronym}[JSONP]\itemsep0pt
	\acro{ecu}[ECU]{European currency unit}
	\acro{css}[CSS]{Cascading Style Sheets}
	\acro{js}[JS]{JavaScript}
	\acro{json}[JSON]{JavaScript Object Notation}
	\acro{html}[HTML]{Hypertext Markup Language}
	\acro{gui}[GUI]{Graphical User Interface}
	\acro{www}[WWW]{World Wide Web}
	\acro{cors}[CORS]{Cross Origin Resource Sharing}
	\acro{api}[API]{afaf}
	\acro{rest}[REST]{Representational State Transfer}
\end{acronym}
	\newpage
	
	% Symbolverzeichnis
	\addcontentsline{toc}{section}{Symbolverzeichnis}
	\input{symb}
	\printnomenclature
	\newpage
	
	%%%%%%%%%%%%%%%%%%%%%%%%%%%%%%%%%%%%%
	%%% HAUPTTEIL %%%%%%%%%%%%%%%%%%%%%%%
	%%%%%%%%%%%%%%%%%%%%%%%%%%%%%%%%%%%%%
	
	\pagestyle{contents}
	
	\renewcommand{\sectionmark}[1]{\markright{#1}}
	\lhead{\fancyplain{}{\rightmark }}
	%\fancyhead[L]{\nouppercase{\leftmark}}
	\pagenumbering{arabic}
	
	\section{Abstract}
% 21cm - 3cm - 2.5cm = 15.5cm textwidth
% Note: It may be necessary to compile the document several times to get a multi-page table to line up properly
	\newpage
	
	\section{Einleitung}

% 21cm - 3cm - 2.5cm = 15.5cm textwidth
% Note: It may be necessary to compile the document several times to get a multi-page table to line up properly

%%%%%%%%%%%%%%%%%%%%%%%%%%%%%%%%%%%%%
%%% TESTS %%%%%%%%%%%%%%%%%%%%%%%%%%%
%%%%%%%%%%%%%%%%%%%%%%%%%%%%%%%%%%%%%
\begin{longtable}[c]{p{4.5cm} p{4.5cm} p{4.5cm}}
	\caption{Diese Tabelle zeigt etwas}
	\label{tab:my-table}\\
	\toprule
	Überschrift1 & Überschrift1 & Überschrift1 \\
	\midrule
	\endfirsthead
	%
	\multicolumn{3}{c}%
	{{\bfseries Tabelle \thetable\ von letzter Seite fortgesetzt}}\\
	\toprule
	Überschrift1 & Überschrift1 & Überschrift1 \\
	\midrule
	\endhead
	%
	Text3 & Text3 & Text3 \\
	Text3 & Text3 & Text3 \\
	\bottomrule
\end{longtable}

%\image{
%	abb/logo_htw.png,
%	fig:something,
%	some caption
%}

\begin{figure}[H]
	\label{fig:something}
	\centering
	\includegraphics[width=1\textwidth, keepaspectratio=true]{abb/logo_htw.png}
	\caption{some caption}
\end{figure}

Text \cite{testblabla}

Beispiel für Quellcode Listings
\begin{lstlisting}[style=customC, caption={Custom C Code}, label={lst:ccode}]
#include <stdio.h>
#define DEBUG
/* Block
* comment */

int main()
{
int i;
// Line comment.
puts("Hello world!");
for (i = 0; i < N; i++)
{
puts("LaTeX is also great for programmers!");
}
return 0;
}
\end{lstlisting}
	\newpage
	
	\section{Hauptteil Eins}
\newpage
\section{Hauptteil Zwei}
	\newpage
	
	\input{pages/4_zusammenfassung}
	\newpage
	
	%%%%%%%%%%%%%%%%%%%%%%%%%%%%%%%%%%%%%
	%%% LITERATURVERZEICHNIS %%%%%%%%%%%%
	%%%%%%%%%%%%%%%%%%%%%%%%%%%%%%%%%%%%%
	
	\addcontentsline{toc}{section}{Literaturverzeichnis}
	\renewcommand\refname{Literaturverzeichnis}
	\printbibliography
	\newpage
	
	%	% Anhang
	%	% \newpage
	%	% \addcontentsline{toc}{section}{Anhang}
	%	% \fancyhead[L]{Anhang} %Kopfzeile links
	%	% \input{anhang/anhang}
	%	
	% Eidesstattliche Erklärung
	\addcontentsline{toc}{section}{Eidesstattliche Erklärung}
	\input{pages/eid_erkl}
	
\end{document}