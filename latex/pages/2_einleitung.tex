\section{Einleitung}

% 21cm - 3cm - 2.5cm = 15.5cm textwidth
% Note: It may be necessary to compile the document several times to get a multi-page table to line up properly

%%%%%%%%%%%%%%%%%%%%%%%%%%%%%%%%%%%%%
%%% TESTS %%%%%%%%%%%%%%%%%%%%%%%%%%%
%%%%%%%%%%%%%%%%%%%%%%%%%%%%%%%%%%%%%
\begin{longtable}[c]{p{4.5cm} p{4.5cm} p{4.5cm}}
	\caption{Diese Tabelle zeigt etwas}
	\label{tab:my-table}\\
	\toprule
	Überschrift1 & Überschrift1 & Überschrift1 \\
	\midrule
	\endfirsthead
	%
	\multicolumn{3}{c}%
	{{\bfseries Tabelle \thetable\ von letzter Seite fortgesetzt}}\\
	\toprule
	Überschrift1 & Überschrift1 & Überschrift1 \\
	\midrule
	\endhead
	%
	Text3 & Text3 & Text3 \\
	Text3 & Text3 & Text3 \\
	\bottomrule
\end{longtable}

%\image{
%	abb/logo_htw.png,
%	fig:something,
%	some caption
%}

\begin{figure}[H]
	\label{fig:something}
	\centering
	\includegraphics[width=1\textwidth, keepaspectratio=true]{abb/logo_htw.png}
	\caption{some caption}
\end{figure}

Text \cite{testblabla}

Beispiel für Quellcode Listings
\begin{lstlisting}[style=customC, caption={Custom C Code}, label={lst:ccode}]
#include <stdio.h>
#define DEBUG
/* Block
* comment */

int main()
{
int i;
// Line comment.
puts("Hello world!");
for (i = 0; i < N; i++)
{
puts("LaTeX is also great for programmers!");
}
return 0;
}
\end{lstlisting}